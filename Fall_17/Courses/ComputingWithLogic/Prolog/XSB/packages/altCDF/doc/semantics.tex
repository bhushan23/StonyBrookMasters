\chapter{The Meaning of Type-0 CDF Instances}
\label{chap:type0} 

%-------------------------------------------------------------------------------
\mycomment{
\subsection{Terminology}

In this paper we use standard logic programming terminology and
notation (e.g. \cite{Lloy84}), extended with the well-founded
semantics~\cite{VGR99}. In addition, we commonly make use of notation
for sentences and formulas in first-order logic.  While most of this
notation is readily apparent, we sometimes write a first-order formula
$\phi$ as $\phi[\overline{X}]$, where $\overline{X}$ is a {\em
superset} of the free variables in $\phi$.
}
%-------------------------------------------------------------------------------

Facts in both Type-0 and Type-1 CDF instances are closely related to
class expressions in description logics.  However, because CDF often
stores class expressions as Prolog facts in an unconventional way, and
because description logics may not be familiar to a logic programming
audience, we present here a somewhat formal introduction to how CDF
represents knowledge.  Users without a mathematical background can
ignore the various axioms and formal definitions that are presented in
this chapter.  Our approach is to introduce a semantics of CDF based
on a translation of a {\em CDF instance} into a set of first-order
logic sentences that constitute an {\em Ontology Theory} whose models
are the models of a CDF instance.  For simplicity of presentation the
description of CDF instances in this section omits certain details
about components, extensional facts and intenstional rules, and other
topics that will be introduced in later chapters.

We illustrate aspects of CDF by means of an example drawn from
electronic commerce.  Health care organizations, such as hospitals,
clinics, etc., have difficulties in buying disposable medical devices
such as sutures, bandages, gloves, and so on.  The difficulty arises
from the fact that these devices may be quite specialized, for intance
when they are used in surgery.  At the same time, since these devices
are disposable, they may need to be purchased frequently.  We consider
concretly the class of {\em absorbable sutures}, which are used for
stitching and securing tissues, and which can be absorbed by the human
body.  Information below is adapted from he U.S. Defence Logistics
Information Service \url{http://www.dlis.mil}, from the Universal
Standard Products and Services Classification~\cite{UNSPSC}, as well
as from websites of various commercial medical supply companies.

\section{Type-0 CDF Instances}
\label{sec:type0} 

We begin with the syntax of Type-0 instances~\footnote{The syntax for
identfiers differs in the actual CDF implementation. See
Section~\ref{sec:instance}.}: 

\begin{definition}[Type-0 Instances: Semantic Level] \label{def:ids}
A {\em Type-0 CDF instance} is a finite set of ground facts for the
predicates \pred{isa/2}, \pred{hasAttr/3}, \pred{allAttr/3},
\pred{classHasAttr/3}, \pred{minAttr/4}, and \pred{maxAttr/4}.  An {\em
identifier} is either a constant or a term.  The arguments of these
predicates are {\em concrete identifiers}, where a term $T$ is an
identifier iff $T$ has the functor symbol {\tt cid/1}, {\tt oid/1},
{\tt rid/1}, or {\tt crid/1} whose argument is either

\begin{enumerate}
\item a constant; or 
\item a term $f(I_1,\ldots,I_n)$ where $I_1,\ldots,I_n$ are identifiers.
\end{enumerate}
In the first case, an identifier is called {\em atomic}; in the second
it is called a {\em product identifier}.
\end{definition}

Despite the simple syntax of Type-0 CDF instances, their semantics
differs from the usual semantics assigned to facts in Prolog.
Identifiers identify sets of objects, or binary relations between
objects.  Furthermore, the facts of a Type-0 CDF instance can
implicitly denote inheritance of various relationships among classes
and objects, as well as inheritance constraints about what
relationships are allowed.

\subsection{Simple Taxonomies in CDF}

\begin{example} \rm \label{ex:suture1}
The following CDF instance illustrates a fragment of a taxonomy for
medical equipment.
%-------------------------------------------
{\tt  {\small 
\begin{tabbing}
foo\=foo\=foo\=foo\=foo\=foo\=foooo\=foooooooooooooooo\=\kill
isa(cid(medicalEquipment),cid('CDF Classes'))  \\
\> isa(cid(woundCareProducts),cid(medicalEquipment)) \\
\> \> \>  isa(cid(suturesAndRelatedProducts),cid(woundCareProducts)) \\
\> \> \> isa(cid(sutures),cid(suturesAndRelatedProducts))  \\
\> \> \> \> isa(cid(absorbableSutures),cid(sutures))  \\
\> \> \> \> isa(cid(nonAbsorbableSutures),cid(sutures)) \\
\> \\
\> \> \>   isa(oid(sutureU245H),cid(absorbableSutures))  \\
\> \> \> \> \>   isa(oid(suture547466),cid(sutures)) 
\end{tabbing} }} 
%-------------------------------------------
\noindent
In CDF, sets of objects are termed {\em classes} to stress the
informality of its sets from the perspective of set theory, and class
identifiers have the functor {\tt cid/1}.  One can read the fact
\begin{verbatim}
      isa(cid(nonAbsorbableSutures),cid(sutures))
\end{verbatim}
as ``all elements in the class {\tt cid(nonAbsorbableSutures)} are
also in the class {\tt cid(sutures)}'' --- in other words, that {\tt
cid(nonAbsorbableSutures)} is a subclass of {\tt cid(sutures)}.
Object identifiers have the functor {\tt oid/1}, and denote classes
with cardinality 1, or {\em singleton classes}.  The fact
\begin{verbatim}
      isa(oid(sutureU245H),cid(absorbableSutures))
\end{verbatim}
can be read as ``the element of the singleton class {\tt
oid(sutureU245H)} is in the class {\tt cid(absorbableSutures)}''.
Note that {\tt cid(absorbableSutures)} is (potentially) more specific
than the class {\tt cid(sutures)}, to which {\tt oid(suture547466)}
belongs.  The class {\tt cid('CDF Classes')} is taken to contain all
objects in the domain of discourse.  All identifiers in this simple
taxonomy are atomic.

The decision of whether to denote an entity as an object or as a class
depends on the use of a given CDF instance.  Here, a given part number
can specify a number of physical parts, but the physical parts are
taken to be identical for the purposes of this instance.  However, if
we were constructing a CDF instance for warehouse management, the
above objects might be better represented as classes, and the physical
objects represented as CDF objects.
\end{example} 

Implicit in the above example is the fact that we use the term {\em
object identifiers} or {\em objects} to denote singleton classes,
class identifiers to denote all classes (including singleton classes).
Elements cannot be denoted directly by CDF facts, but only through
singleton classes that contain them (and are isomorphic to them).  At
this point, we can begin defining the semantics of Type-0 CDF
instances.

%-------------------
\begin{definition} \label{def:ontolang}
\index{ontology language} \index{ontology structure} \index{ontology
theory}
%
An {\em ontology language} is a first-order language with equality
containing the predicates: {\em isClass/1, isElt/1, isRel/1, isCrel/1,
elt/2, rel/3, and crel/3}, and a countable set of constants.  An {\em
ontology structure} is a structure defined over an ontology language.
An {\em ontology theory} is a set of first-order sentences formed over
an ontology language that includes a set of {\em core axioms}, defined
below, along with the defined predicate:
\[
isObj(X) =_{def} isClass(X) \wedge 
	((elt(E_1,X) \wedge elt(E_2,X)) \Rightarrow E_1 = E_2)
\]
If $\cT$ is an ontology theory formed over an ontology
language $\cL$, an ontology structure $S$ over $\cL$ is a model of
$\cT$ if every sentence of $\cT$ is satisfied in $S$.
\end{definition}
%-------------------

\index{sorting predicates}
By convention we assume that the variables in an ontology language are
indexed by the set of natural numbers.  In this paper we will restrict
our attention to ontology languages whose constant and function
symbols are identifiers as described in Definition~\ref{def:ids}.
Informally $isClass/1$ indicates that an identifier $I$ is a class
name or {\em class identifier}; $isElt/1$ indicates that an identifier
$I$ is an element of a class; $isRel/1$ that $I$ is a {\em relation
identifier}; and $isCrel/1$ that $I$ is a {\em class-relation
identifier}; and $isObj/1$ that $I$ is an {\em object identifier}.  We
sometimes call these 5 predicates {\em sorting predicates}.
$elt(E,C)$ indicates that an element $O$ is a member of class
identifier $C$; $rel(O_1,R,O_2)$ indicates that an element $E_1$ has a
$R$ relation to an element $E_2$; and $crel(C_1,R,E)$ indicates that
the class identifier $C_1$ has a $R$ relation to an object identifier
$E$.

The first core axiom ensures that objects, classes, relations and
class-relations all have distinct identifiers within an ontology
language.
%-----------
\begin{axiom}[Distinct Identifiers] \label{ax:distinct}
\index{axioms!distinct identifiers} 
\ \\
\begin{tabbing}
foo\=foo\=foo\=foo\=foo\=foo\=foooo\=foooooooooooooooo\=\kill
\> $\neg \exists Id [isClass(Id) \wedge (isElt(Id) \vee isRel(Id) 
	                                 \vee isCrel(Id))] \wedge $ \\
\> $\neg \exists Id [isObj(Id) \wedge (isElt(Id) \vee isCrel(Id))] \wedge $ \\
\> $\neg \exists Id [isElt(Id) \wedge isCrel(Id)] $ 
\end{tabbing}
\end{axiom} 
%-----------

$isClass/1$, $isElt/1$, $isRel/1$, and $isCrel/1$ provide an effective
sorting that extends to all predicates, as the next axiom indicates.

\begin{axiom}[Predicate Sorts] \rm \label{ax:sorts}
\index{axioms!predicate sorts} 
\ \\
\begin{tabbing}
foo\=foo\=foo\=foo\=foo\=foo\=foooo\=foooooooooooooooo\=\kill
\> \> $(\forall X,Y) [elt(X,Y) \Rightarrow (isElt(X) \wedge isClass(Y))]
								\wedge $ \\
\> \> $(\forall X,Y,Z) [rel(X,Y,Z) \Rightarrow (isElt(X) \wedge isRel(Y)
					   \wedge isElt(Z))] \wedge  $ \\
\> \> $(\forall X,Y,Z) [crel(X,Y,Z) \Rightarrow (isClass(X) \wedge isRel(Y)
					   \wedge isElt(Z))] $ \\
\end{tabbing}
\end{axiom} 

%-----------

The following definition of $IdSort$ relates the functors of
identifiers in a Type-0 CDF instance to their sort in an ontology
theory.  It will be used in the various instance axioms to enforce
proper sorting of product identifiers.

\begin{definition}{\bf [IdSort]} \label{def:IdSort}
Let {\em I} is be an identifier. Then {\em IdSort(I)} is equal to {\em
isClass(I)} if the main functor symbol of {\em I} is {\tt cid/1}; {\em
isObj(I)} if the main functor symbol of {\em I} is {\tt oid/1}; {\em
isRel(I)} if the main functor symbol of {\em I} is {\tt rid/1}; and
{\em isCrel(I)} if the main functor symbol of {\em I} is {\tt crid/1}.
\end{definition}

%-----------------------------------------------------------------------------------------
\mycomment{
\begin{definition}{\bf [IdSort]}  Let {\em I} is be an identifier, and
$\cI$ be the set of identifiers occurring in {\em I}. Then
%
\[ IdSort(I) =_{def} \bigwedge_{I' \in \cI} Sort(I') \]
%
where {\em Sort(I)} is equal to {\em isClass(I)} if the main functor
symbol of {\em I} is {\tt cid/1}; {\em isObj(I)} if the main functor
symbol of {\em I} is {\tt oid/1}; {\em isRel(I)} if the main functor
symbol of {\em I} is {\tt rid/1}; and {\em isCrel(I)} if the main
functor symbol of {\em I} is {\tt crid/1}.
\end{definition}
}
%-----------------------------------------------------------------------------------------

From Definitions~\ref{def:IdSort} and~\ref{def:ontolang}, it is easy
to see that for any object identifier $O$, $IdSort(O) = isObj(O)$, and
$isObj(O) \Rightarrow isClass(O)$.

%-----------
\begin{instance} [Translation of {\tt isa/2}] \rm 
For each fact of the form {\tt isa(Id$_1$,Id$_2$)} add the axiom
\ \\
\begin{tabbing}
foo\=foo\=foo\=foo\=foo\=foo\=foooo\=foooooooooooooooo\=\kill
\> $ IdSort(Id_1) \wedge IdSort(Id_2) \wedge $ \\

\> \> $ (((isClass(Id_1) \wedge isClass(Id_2)) \wedge
	(\forall X) [elt(X,Id_1) \Rightarrow elt(X,Id_2)]) \vee $ \\

%\> \> $ ((isObj(Id_1) \wedge isClass(Id_2)) \wedge elt(Id_1,Id_2)) \vee $ \\

\> \> $ ((isRel(Id_1) \wedge isRel(Id_2)) \wedge
	(\forall X,Y)[rel(X,Id_1,Y) \Rightarrow rel(X,Id_2,Y)]) \vee $ \\

\> \> $ ((isCrel(Id_1) \wedge isCrel(Id_2)) \wedge
	(\forall X,Y)[crel(X,Id_1,Y) \Rightarrow crel(X,Id_2,Y)])) $ 
\end{tabbing}
denoted as {\tt isa(Id$_1$,Id$_2$)}$^{\cI}$.
\end{instance}
%-----------

Note that the reflexive and transitive closure of {\tt isa/2} is an
immediate consequence of its translation rule.  The next axiom is
technical.  It is important for the semantics of relations that each
class have at least one member.
%-----------
\begin{axiom}[Non-Empty Classes] \label{ax:nonnull}
\index{axioms!non-empty classes} 
\[ (\forall X) [isClass(X) \Rightarrow (\exists Y) [elt(Y,X)]] \]
\end{axiom} 
%-----------

The last core axiom for these predicates ensures is that each class is
a subclass of {\tt cid('CDF Classes')}.

%-----------
\begin{axiom}[Domain Containment] \label{ax:contained}
\index{axioms!domain containment} 
\[ (\forall X) [isElt(X) \Rightarrow elt(X,cid('CDF\ Classes'))] \]
\end{axiom} 
%-----------

\subsection{General Relations between Objects in  Classes}

\begin{example} \rm \label{ex:suture2}
The class {\tt cid(sutures)} can be further defined by its relations
to other classes.  For instance, an object in {\tt cid(sutures)} may
have a designation of its needle design indicating whether it is to be
used for abdominal surgeries, thoracic surgeries, or other purposes.
This information is indicated by the following CDF facts:
%
{\small
\begin{tabbing}
fooo\=foo\=foo\=foo\=fooooooooooooooooooooooooooooooo\=ooooooooooooo\=\kill
\> {\tt isa(rid(hasNeedleDesign),rid('CDF Relations'))} \\
\\
\> {\tt isa(cid(domainTypes),cid('CDF Classes'))} \\
\> \> {\tt isa(cid(needleDesignTypes),cid(domainTypes))} \\
\> \> \> {\tt isa(cid(Abdominal),cid(needleDesign))} \\
\> \> \> {\tt isa(cid(Abscisson),cid(needleDesign))} \\
\> \> \> {\tt isa(cid('Adson Dura'),cid(needleDesign))} \\
\> {\it \% 126 other values..}  \\
\\
\> {\tt allAttr(cid(sutures),rid(hasNeedleDesign),cid(needleDesign))} 
\end{tabbing}
}
%
\noindent
The {\tt allAttr/3} fact above can be read as ``if any object in {\tt
cid(absorbableSutures)} has a {\tt rid(hasNeedleDesign)} relation, it
must be to an object in the class {\tt cid(needleDesign)}''.  That
{\tt hasNeedleDesign} is a relation is indicated by clothing it in the
functor {\tt rid/1}.  This relation is an immediate subclass of all
{\tt CDF Relations} which in turn is taken to represent the universal
binary relation over the domain of discourse.  Thus the {\tt
allAttr/3} fact effectively types the range of {\tt
rid(hasNeedleDesign)} relations, stemming from objects in the class
{\tt cid(absorbableSutures)}, but it does not indicate the existence
of such a relationship.  From a user's point of view, the {\tt
rid(hasNeedleDesign)} relation can be thought of as an optional
attribute for a given {\tt cid(absorbableSutures)} object.  Sample
values for {\tt cid(hasNeedleDesignTypes)} are also given.
\end{example} 

%-------------------

{\tt allAttr/3} provides a simple but powerful mechanism for
inheritance of typing among CDF classes:
%---------------------------------------------------------------------------
\mycomment{
, as can be seen from the
following translation rule, which uses the defined formula

\[ 
elt(X,Y) =_{def} ((isClass(Y) \wedge elt(X,Y)) \vee (isObj(Y) 
			\wedge X = Y))
\] }
%---------------------------------------------------------------------------

\begin{instance} [Translation of {\tt allAttr/3}] \rm 
For each fact of the form {\tt allAttr(Id$_1$,Rid,Id$_2$)} add the instance
axiom: 
\begin{tabbing}
foo\=foo\=foo\=foo\=foo\=foo\=foooo\=foooooooooooooooo\=\kill
\> $ IdSort(Id_1) \wedge IdSort(Rid) \wedge IdSort(Id_2) \wedge $ \\
\> \> $ IsClass(Id_1) \wedge IsRel(Rid) \wedge IsClass(Id_2) \wedge $ \\
\> \> $(\forall X, Y) [(elt(X,Id_1) \wedge rel(X,Rid,Y))
					\Rightarrow elt(Y,Id_2)] $
\end{tabbing}
denoted as {\tt allAttr(Id$_1$,Rid,Id$_2$)$^{\cI}$}.
\end{instance}

%----------------------
\mycomment{
\begin{tabbing}
foo\=foo\=foo\=foo\=foo\=foo\=foooo\=foooooooooooooooo\=\kill
\> $ IdSort(Id_1) \wedge IdSort(Rid) \wedge IdSort(Id_2) \wedge $ \\
\> \> $ (IsClass(Id_1) \vee isObj(Id_1)) \wedge IsRel(Rid) \wedge 
	 (IsClass(Id_2) \vee isObj(Id_2)) \wedge $ \\
\> \> $(\forall X, Y) [(elt(X,Id_1) \wedge rel(X,Rid,Y))
					\Rightarrow elt(Y,Id_2)] $
\end{tabbing}
}
%----------------------
\begin{example} \rm \label{ex:hasAttr}
While {\tt allAttr/3} indicates a typing for relations, it does not
indicate that a relation must exist for elements of a class.  This
statement is made by \pred{hasAttr/3}.  The relation {\tt
rid(hasPointStyle)} for the class {\tt cid(absorbableSutures)} is
taken to be required in this schema.  The facts
%
{\small
\begin{tabbing}
fooo\=foo\=foo\=foo\=foooooooooooooooooooooooooooo\=ooooooooooooo\=\kill
\> {\tt allAttr(cid(absorbableSutures),rid(hasPointStyle),cid(pointStyle)) } \\
\> {\tt hasAttr(cid(absorbableSutures),rid(hasPointStyle),cid(pointStyle)) }
\end{tabbing}
}
%
\noindent
indicate not only the range of such relationships, but that such a
relationship must exist.  Indeed, the {\tt hasAttr/3} fact can be read
as ``all objects in the class {\tt cid(absorbableSutures)} have a
relation {\tt rid(hasPointStyle)} to an object in the class {\tt
cid(pointStyle)}''.  The facts below also give information about the
{\tt rid(hasPointStyle)} relation.
%
{\small
\begin{tabbing}
fooo\=foo\=foo\=foo\=foooooooooooboooooooooooooooo\=ooooooooooooo\=\kill
\> {\tt isa(cid(pointStyle),cid(domainTypes))} \\
\> {\tt isa(cid(regularCuttingEdge),cid(pointStyle))} \\
\> {\tt isa(cid(reverseCuttingEdge),cid(pointStyle))} \\
\> {\tt isa(cid(scalpelPoint),cid(pointStyle))} \\
\> {\it \% 10 other values.} \\
\\
\> {\tt hasAttr(oid(sutureU245H),rid(hasPointStyle),cid(regularCuttingEdge))}
\end{tabbing}
} 
\noindent
The last of the above facts can be read as ``the object {\tt
oid(sutureU245H)} has a {\tt rid(hasPointStyle)} relation to an object
in the class {\tt cid(pointStyle)}''.  
\end{example}

Not surprisingly, the definition of {\tt hasAttr/3} bears some
similarity to that of {\tt allAttr/3}.

\begin{instance} [Translation of \pred{hasAttr/3}] \rm 
For each fact of the form {\tt hasAttr(Id$_1$,Rid,Id$_2$)} add the instance
axiom: 
\begin{tabbing}
foo\=foo\=foo\=foo\=foo\=foo\=foooo\=foooooooooooooooo\=\kill
\> $ IdSort(Id_1) \wedge IdSort(Id_2) \wedge IdSort(Id_2) \wedge $ \\
\> \> $ IsClass(Id_1) \wedge IsRel(Rid) \wedge
	 IsClass(Id_2) \wedge $ \\
\> \> \> $ (\forall X) [elt(X,Id_1) \Rightarrow \exists Y [rel(X,Rid,Y) 
					\wedge elt(Y,Id_2)]]$
\end{tabbing}
denoted as {\tt hasAttr(Id$_1$,Rid,Id$_2$)$^{\cI}$}.
\end{instance}

We next turn to relational axioms that indicate the cardinality of
various relations.

\begin{example} \rm  \label{ex:maxAttr}
For our purposes, an object in {\tt cid(absorbableSutures)} can be
thought of as consisting of a needle and a thread~\footnote{The thread
is often called a suture.  We are assuming for purposes of
illustration that all sutures are --- in suture-speak --- ``armed''.}.  This is
represented by the facts:
%
{\small
\begin{tabbing}
foo\=foo\=foo\=foo\=foo\=foo\=foooo\=foooooooooooooooo\=\kill
\> {\tt allAttr(cid(absorbableSutures),rid(hasImmedPart),cid(absSutPart)) } \\
\> {\tt hasAttr(cid(absorbableSutures),rid(hasImmedPart),cid(absSutNeedle)) } \\
\> {\tt hasAttr(cid(absorbableSutures),rid(hasImmedPart),cid(absSutThread)) } \\
\\
\> {\tt isa(cid(absSutPart),cid(suturesAndRelatedProducts))}  \\
\> \> {\tt isa(cid(absSutNeedle),cid(absSutPart)) } \\
\> \> {\tt isa(cid(absSutSuture),cid(absSutPart)) } 
\end{tabbing}
}
%
\noindent
A needle for an absorbable suture is typically made of a different
material than the thread to which the needle is attached.  Each of
these materials may be important in choosing an absorbable suture, and
each of these materials are unique.  The facts
%
{\small
\begin{tabbing}
foo\=foo\=foo\=foo\=foo\=foo\=foooo\=foooooooooooooooo\=\kill
\> {\tt hasAttr(cid(absSutPart),rid(hasMaterial),cid(absSutMaterial)) } \\
\> {\tt maxAttr(cid(absSutPart),rid(hasMaterial),cid(absSutMaterial),1) } \\
\\
\> {\tt isa(cid(material),cid(domainTypes))} \\
\> {\tt isa(cid(absSutMaterial),cid(material)} \\
\> {\tt isa(cid(gut),cid(absSutMaterial))} \\
\> {\tt isa(cid(polyglyconate),cid(absSutMaterial))} \\
\> {\tt isa(cid(polyglyconicAcid),cid(absSutMaterial)) } 
\end{tabbing}
}
%
\noindent
indicate that each {\tt cid(absSutPart)} has a unique material.  The
{\tt maxAttr/4} fact can be read as ``Each object in the class {\tt
cid(absSutPart)} has at most 1 {\tt rid(hasMaterial)} relation to an
object in the class {\tt cid(absSutMaterial)}''.
\end{example}

In order to define the semantics of {\tt maxAttr/4}, let
$\exists^{\leq n}X_m[ \phi(X,Z)]$ be defined as an abbreviation for
the formula
\[
  \exists x_m,...,\exists x_{m+n} [\bigwedge_{m \leq i \leq m+n} 
	\phi(x_i,\overline{z}) 
	\Rightarrow \bigvee_{m \leq i < j \leq m+n} x_i  = x_j]
\]
i.e., for the formula indicating that there are at most $N$ non-equal
elements satisfying $\phi(x,z)$.  The abbreviation $\exists^{\geq N}$
is defined similarly to indicate that a formula is satisfied by at
least $N$ non-equal elements.

\begin{instance} [Translation of {\tt maxAttr/4}] \rm 
For each fact of the form {\tt maxAttr(Id$_1$,Rid,Id$_2$,N)} add the instance
axiom: 
\begin{tabbing}
foo\=foo\=foo\=foo\=foo\=foo\=foooo\=foooooooooooooooo\=\kill
\> $ IdSort(Id_1) \wedge IdSort(Id_2) \wedge IdSort(Id_2) \wedge $ \\
\> \> $ IsClass(Id_1) \wedge IsRel(Rid) \wedge
	 (IsClass(Id_2) \wedge $ \\
\> \> \> $ (\forall X) [elt(X,Id_1) \Rightarrow \exists^{\leq N} Y [rel(X,Rid,Y) 
					\wedge elt(Y,Id_2)]]$
\end{tabbing}
denoted as {\tt maxAttr(Id$_1$,Rid,Id$_2$,N)$^{\cI}$}.
\end{instance}

A corresponding predicate, {\tt minAttr/4} is used to indicate a
minimality restriction on a relation.  {\tt minAttr/4} is defined
similarly to {\tt maxAttr/4}, but using $\exists^{\geq N}$ rather than
$\exists^{\leq N}$.  Indeed, the predicate
%
{\small {\tt 
\begin{tabbing}
foo\=foo\=foo\=foo\=foo\=foo\=foooo\=foooooooooooooooo\=\kill
\> hasAttr(cid(absSutPart),rid(hasMaterial),cid(absSutMaterial)).
\end{tabbing}
} }
%
\noindent
could be replaced by the predicate
%
{\small {\tt 
\begin{tabbing}
foo\=foo\=foo\=foo\=foo\=foo\=foooo\=foooooooooooooooo\=\kill
\> minAttr(cid(absSutPart),rid(hasMaterial),cid(absSutMaterial),1).
\end{tabbing}
} }
%
\subsection{Class Relations}
Each of the predicates discussed so far are inheritable in their first
argument.  For instance, the fact
{\small {\tt 
\begin{tabbing}
foo\=foo\=foo\=foo\=foo\=foo\=foooo\=foooooooooooooooo\=\kill
\> hasAttr(cid(absSutPart),rid(hasMaterial),cid(absSutMaterial)).
\end{tabbing}
} } 
%
\noindent
implies that every subclass of {\tt cid(absSutures)} will have a
material in the class {\tt cid(absSutMaterial)}.  However, classes may
have relations that do {\em not} hold for their subclasses or members.
For instance, a finite set may have a given cardinality, but its
proper subsets will have a different cardinality.  Such relations are
termed {\em class relations}.

%-------------------
\begin{example} \label{ex:strel} \rm 
A practical example of a class relation comes from an application that
may be called part equivalency matching.  In this application, the
possible attributes for a class of parts are given various weights.
Two parts match if the sum of the weights of their attributes that
match are above a given threshold.  The weighting for the {\tt
cid(pointStyle)} of sutures might be given as:
%----------------------------------
{\small 
{\tt 
\begin{tabbing}
foo\=foo\=foo\=foo\=foo\=foo\=foooo\=foooooooooooooooo\=\kill
\> isa(cid(pointStyleWeight),cid('CDF Classes')) \\
\> \> isa(cid(highWeight),cid(pointStyleWeight)) \\
\> \> isa(cid(lowWeight),cid(pointStyleWeight)) \\
\\
\> classHasAttr(cid(sutures),crid(pointStyleWeight),cid(highWeight))
\end{tabbing}
} }
%----------------------------------
\noindent
The {\tt classHasAttr/3} fact can be read as ``the class {\tt
cid(sutures)} has a {\tt crid(pointStyleWeight)} relation to an object
in {\tt cid(highWeight)}.  Matching weights are made non-inheritable
via {\tt classHasAttr/3} because a weight may depend on a given
classification of a part.  For instance if a part were classified as a
{\tt cid(nonAbsorbableSuture)}, its {\tt cid(pointStyle)} might weigh
less (or more) for determining whether two sutures are equivalent.
\end{example}
%------------------
\begin{instance} [Translation of {\tt classHasAttr/3}] \rm 
For each fact of the form {\tt classHasAttr(Id$_1$,CRid,Id$_2$)} add the
instance axiom: 
\begin{tabbing}
foo\=foo\=foo\=foo\=foo\=foo\=foooo\=foooooooooooooooo\=\kill
\> $ IdSort(Id_1) \wedge IdSort(CRid) \wedge IdSort(Id_2) \wedge $ \\
\> \> $ IsClass(Id_1) \wedge IsCrel(CRid) \wedge isClass(Id_2) \wedge $ \\
\> \> \> $ (\exists X) [elt(X,Id_2) \wedge crel(Id_1,CRid,X)] $
\end{tabbing}
denoted as {\tt classHasAttr(Id$_1$,Rid,Id$_2$)$^{\cI}$}.
\end{instance}
%----------

