\chapter{Notes and Ideas}

\section{Default Reasoning}

CDF allows various types of reasoning under incomplete information.
To take an example of how CDF supports incomplete information from a
Type-0 level, consider that Examples~\ref{ex:suture1}
to~\ref{ex:hasAttr} imply~\footnote{
In this section we use semantic identifiers.}
{\small {\tt 
\begin{tabbing}
foo\=foo\=foo\=foo\=foo\=foo\=foooo\=foooooooooooooooo\=\kill
\> hasAttr(oid(suture547466),rid(hasPointStyle),cid(pointStyle))
\end{tabbing}
} }
%
\noindent
In other words, that {\tt oid(suture547466)} has a point style, but it
is not known exactly what that point style is.  At the Type-1 level,
the goal
{\small {\tt 
\begin{tabbing}
foo\=foo\=foo\=foo\=foo\=foo\=foooo\=foooooooooooooooo\=\kill
\> ?-
consistentWith(oid(suture547466),exists(rid(hasPointStyle),cid(scalpelPoint))).
\end{tabbing}
} }
%
\noindent
will succeed, indicating that statement ``{\tt oid(suture547466)}
has a point style of {\tt cid(scalpelpoint)}'' is consistent with the
CDF instance, although not implied by it.  

In many cases, it may be useful to make a statement like {\em If a
{\tt cid(absorbableSuture)} has no known point style, then assume that
its point style is a {\tt cid(scalpelPoint)''}}.  Such a statement can
be seen as a type of non-monotonic inference, and there have been many
proposals to add various sorts of non-monotonic reasoning to ontology
systems using default logic~\cite{BaaH95,BaaH95a} or non-monotonic
negation~\cite{KiLW95}.  

Given the design of CDF there are various considerations to make for
adding non-monotonic reasoning.  On the one hand, the use of
intensional rules, based on XSB's implementation of well-founded
semantics gives a fairly general approach to implementing
non-monotonic reasoning.  To make a statement like the one above, a
user could add the rules: 
{\small {\tt 
\begin{tabbing}
foo\=foo\=foo\=foo\=foo\=foo\=foooo\=foooooooooooooooo\=\kill
\> hasAttr\_int(X,rid(hasPointStyle),cid(scalpelpoint)) :-  \\
\> \> isa(X,cid(absorbableSuture)), \\
\> \> tnot(hasNonScalpelPoint(X)). \\
\> \\
\> hasNonScalpelPoint(X):-  \\
\> \> 	hasAttr(X,rid(hasPointStyle),Point),  \\
\> \> 	Point \== cid(scalpelPoint).
\end{tabbing}
} }
%
\noindent
There are several problems with such an approach.  First,
implementation of this simple default rule requires a level of XSB
programming that is beyond most domain experts.  Second, such a rule,
added in an ad hoc manner, would be difficult to integrate into the
CDF prover.  Third, this approach does not take into account the fact
that multiple default rules might be applicable to a given object.

Our approach to default reasoning within CDF is based on the
assumption that the most common use for default reasoning within a
desription logic is to allow default properties to be assigned to
objects in a given class.  Accordingly, we provide the predicate
{\small {\tt 
\begin{tabbing}
foo\=foo\=foo\=foo\=foo\=foo\=foooo\=foooooooooooooooo\=\kill
\> defaultHasAttr(Source,Relation,Target)
\end{tabbing}
} }
%
\noindent
The predicate {\tt immed\_hasAttr/3} (Section~\ref{sec:instance}
then have a third clause added to it as follows:
{\small {\tt 
\begin{tabbing}
foo\=foo\=foo\=foo\=foo\=foo\=foooo\=foooooooooooooooo\=\kill
\> immed\_hasAttr(Source,Relation,Target):-  \\ 
\> \> 	defaultHasAttr(Source,Relation,Target), \\
\> \> 	tnot(conflictingHasAttr(Source,Relation,Target)). \\
\> \\
\> conflictingHasAttr(Source,Relation,Target):-  \\
\> \> hasAttr(SourceAlt,RelationAlt,TargetAlt), \\
\> \> isa(SourceAlt,Source), \\
\> \> isa(Relation,RelationAlt), \\
\> \> not isa(Target,TargetAlt).
\end{tabbing}
} }
%
Thus {\tt defaultHasAttr(Source,Relation,Target)} would take effect if
there were no {\tt hasAttr/3} relation (possibly depending on a
separate default relation) with a more specific source, a more general
relation, and a target that was not more general than {\tt Target}.
Also note that the addition of default rules may make answers to a
Type-0 query undefined in the well-founded semantics.

{\sc Example -- overriding inheritance based on more specific sources}

{\sc Example -- Nixon diamond leading to 3-valued model.}

{\sc need to work through examples with CDFTP}.

\section{Number Restrictions}

${\cal ALC}(\circ)$ can be seen to be decidable by reduction to
$\cL^2$.  Consider the formula $exists(R \circ S \circ T,C)$ which
translates to 
\[
\exists X_1,X_2,X_3 [R(X_0,X_1) \wedge S(X_1,X_2) \wedge T(X_2,X_3)
	\wedge C(X_3)
\]
Which is equivalent to 
\[
\exists X_1 [R(X_0,X_1) \wedge \exists X_2 [S(X_1,X_2) \wedge 
	\exists X_3 [T(X_2,X_3) \wedge C(X_3) ] ] ]
\]
which is equivalent to 
\[
\exists X_1 [R(X_0,X_1) \wedge \exists X_0 [S(X_1,X_0) \wedge 
	\exists X_1 [T(X_0,X_1) \wedge C(X_1) ] ] ]
\]
Such a proof, extended to induction on formulas would be much simpler
than that provided in \cite{BaaS99}. The proofs of ${\cal
ALCN}(\circ)$, and ${\cal ALCQ}(\circ)$ are similar.

However, consider ${\cal ALC}(\circ,\sqcup)$.  The formula $exists((R
\sqcup T \circ S),C)$ is equivalent to 
\[
\exists X_1,X_2 [(R(X_0,X_1) \vee (S(X_0,X_2) \wedge T(X_2,X_1)))
	\wedge C(X_1)]
\]
which is not in $\cL^2$ as $X_1$ is connected by $S$ to $X_0$ and by
$T$ to $X_1$.  So the question is what subsets of $\cL^3$ are
decidable, and whether they can be expressed in ${\cal
ALC(\circ,\sqcup)}$.  In addition, there is the question of whether
such subclasses differ between $\cC^3$ and $\cL^3$.

\section{2-Variable Logics}

Questions: Assume a Graph whose edges are labelled $R_i$ and whose
nodes, $a \in A$, have properties $P_i$.

\begin{enumerate}
\item Does a have an outgoing $E_i$ edge to a vertex in $P$?
\begin{itemize}
\item ML/ALC:  \[ (\cA,a) \models \diamond_i P \]
\item $\cL^2$: \[ (\cA,a) \models 
	\exists X_1 [ R(a,X_1) \wedge P(X_1) ] \]
\end{itemize}

\item Do all vertices that can be reached from a by traversing one
$E_2$ edge have the above property?
\begin{itemize}
\item ML/ALC:  \[ (\cA,a) \models \square_2 \diamond_i P \]
\item $\cL^2$: \[ (\cA,a) \models \forall X_1 [R_2(a,X_1) \wedge
	( \exists X_2 [ R_i(X_1,X_2) \wedge P(X_2)) ] ] \]
\end{itemize}

\item Is there an $E_i$ path of length 17 from a that ends with a
	vertex in $P$?
\begin{itemize}
\item ML/ALC:  \[ (\cA,a) \models \diamond_i \diamond_i ... \diamond_i P \]
\item $\cL^2$: \[ (\cA,a) \models 
	\exists X_1 [ R_i(a,X_1) \wedge (R_i(X_1,X_2) (\wedge ... 
			(\wedge (R_i(X_{16},X_{17}) \wedge P(X_1) ] \]
\end{itemize}
\item Is there an incoming $E_i$ edge at a
\begin{itemize}
\item ML/ALC: cannot do within the language, although you can extend
			the language to capture this -- need ${\cal ALCI}$
\item $\cL^2$: \[ (\cA,a) \models 
	\exists X_1 [ R_i(X_1,a) \wedge P(X_1) ] \]
\end{itemize}
\item Is there a vertex linked to a by both an $E_1$ edge and an $E_2$
edge? 
\begin{itemize}
\item Need ${\cal ALCQHI}$, where $R_1 \subseteq R \wedge R_2 \subseteq R$
\[ atMost(1,R^-,\top) \wedge exists(R_1^-,\top) \wedge exists(R_2^-,\top)
\]
\item $\cL^2$: \[ (\cA,a) \models 
	\exists X_1 [ R_1(X_1,a) \wedge R_2(X_1,a) ] \]
\end{itemize}
\item Is every vertex in $P$ reachable from a on an E-path of length at most 3.
\item Is it possible to reach $P$ from a on an $E_i$ path?
\item Do all $E_1$ paths from a eventially hit $P_1$ and before hitting
$P_1$ pass only through vertices in $P_2$? 
\end{enumerate}

\paragraph{Pebble Games}
Naturally, FO$^2$-morphisms can be described by a 2-pebble game.  The
idea is that player 2 has a winning strategy for k-move games if two
structures satisfy exactly the same FO2 formulas of quantifier rank k.
If the number of responses for player 2 matches the number of
challenges, it is a bijective pebble game, and captures morphisms for
C2.  If the game is weakened to accessability, it captures morphisms
for modal logic.  That is, ML requires only a 1-pebble game along
accessability relations where E-accessable relations are matched.  And
this is a ... bisimulation.  Graded MLs (ALCQ) requres that the
matching be a bijection.

\paragraph{Other 2 variable logics}
$TC^2$ has a transitive closure operator.  Let $\phi(x,y)$ be a
formula.  Then $[\phi]_{TC}(a,b)$ is true if (a,b) is in the
transitive closure of $(a_1,a_2)$ s.t. $\cM \models \phi(a_1,a_2)$.
While $TC^2$ formulas syntactically have only two variables, it is not
a 2-variable logic in the EF sense.  To see this, note that a
structure consisting of two non-directed graphs each w. 3 vertices is
EF-indistinguishable from a structure consisting of a single
non-directed graph with 6 vertices.  However there are $TC^2$ formulas
that can distinguish between these two.

$FP^2$ is 2-variable fixed-point logic.  I.e. if $\phi(X,x)$ is
positive in $X$, then $[\mu X,x]\phi(X,x)$ is the lfp of the equation
$S = a:phi(S,a)$.  Would be good to have a better definition.

$CL^2$ is $FO^2$ extended with $<>^*$ and []*.  Semantically this is a
restriction of fixed-point logic.  $\psi_1(x) = (<\chi>^*\phi)(x)$ and
its semantics is $\mu X [ \phi \vee <\chi>X]$.  $\psi_2(x) =
([\chi]^*\phi)(x)$, and its semantics is $\mu X [ \phi \vee [\chi]X]$.

Are these logics closed under negation?  It appears that $TC^2 \subset
CL^2$, but I'm not sure.

\[ FO^2 \subset CL^2 \subset FP^2 \]
\[ML \subset F0^2, CTL \subset CL^2, CTL \subset TC^2, MMC \subset FP^2 \]
\[ ML \subset CTL \subset MMC \]

%----------------------------------------------------------------------------------------------
\paragraph{Complexity and Satistfiability of $FO^2$}

To prove decidability (and complexity) of $FO^2$ use the following
reductions.  First, introduce predicates
\[ \psi_i(x) \lra \exists y \phi(y) ; \psi_i(x) \lra \forall y \phi(y) \]

So basically, we keep replacing nested quantifiers by adding new
conjuncts that are equivalences.  This unravels the formula to
$\forall \forall$ and $\forall \exists$ expressions..  This reduction
is P-time and we end up with a sentence of the form
\[
\forall x \forall y \phi_0 \wedge^{m}_{i = 1} \forall x \exists y \phi_i
\]

An atomic (or basic) {\em n-type} is a maximally consistent set of
atomic formulas in at most $n$ variables in a given language L.  Let
$S$ be a structure, then any sequence of individuals in $S$ has a
unique n-type.

Let $\alpha$ be the set of 1 types in $\cL$, and $\beta$ the set of
2-types in $\cL$.  $\alpha_{\cA}$, $\beta_{\cA}$ are the 1/2 types
satisfied in $\cA$.  For a quantifier-free $FO^2$ formula $\phi$,
$\beta_{\phi}$ is the set of 2 types $\beta$ such that $\beta \models
\phi$.

Note that the Scott-Normal Form sentence 
\[
 \forall X_0,X_1 [\phi_0] \wedge_{i = 1}^m \forall X_0 \exists X_1 \phi_i
\]
is true in $\cA$ iff $\beta_{\cA} \subseteq \beta_{\phi_0}$ and for
each $i$, for each individual $a \in A$, there is some $b \not = a$,
such that $type_{\cA}(a,b) \in \beta_{\phi_i}$.  An element $a \in A$
is a {\em king} if its 1-type is unique in $\cA$.

The model construction procedure for $FO^2$ is as follows.  First,
guess the kings in $\cA$.  Next, guess the court for the kings (to
satisfy the $\forall \exists$ formulas.  Then do a circular guessing
for $\forall \exists$ formulas among three boxes of elements
(types). Each of these boxes has $m$ elements a guessed set of
non-royal 1-types.


%----------------------------------------------------------------------------------------------
\paragraph{Complexity and Satistfiability of $C^2$}

An additional transformation is needed.  Transform
$\exists^{>n}x\phi(x)$ in the following manner.  Introduce singleton
sets as follows.
\[ \forall x \vee^n_{i=1} Si(x) \Ra \phi(x) \wedge 
\forall x \wedge_{i \not = j} \neg (S_i(x) \wedge S_j(x)) 
\wedge^n_{i = 1}\exists^{=1} x S_i(x)
\]
This reduction is exponential (if $n$ is binary) and produces
\[
\forall x \forall y \phi_0 \wedge^{m}_{i = 1} \forall x \exists^{=1} y \phi_i
\]

%----------------------------------------------------------------------------------------------
\section{Covers}
\[ all(R,C) = \forall X_1 [R(X_0,X_1) \Ra C(X_1)] \]

\[ exists(R,C) = \exists X_1 [R(X_0,X_1) \wedge C(X_1)] \]

\[ covers(R,C) = \forall X_1 [C(X_1) \Ra R(X_0,X_1)] \]

\section{3-sat and ASP}

Let $\{C_1,...,C_n\}$ be a 3-sat problem where each $C_i$ consists of
literals $L_{i1},L_{i2},L_{i3}$.  Translate this set to an ASP program
$P$
%
\begin{verbatim}
     A :- C_1,...,C_n.

     C_i :- L'_i1.
     C_i :- L'_i2.
     C_i :- L]_i3.
\end{verbatim}

Where $L'_{ij} = L_{ij}$ if $L_{ij}$ is positive and $L'_{ij} =
not\_A_{ij}$ if $L_{ij} = A_{ij}$ for some atom $A_{ij}$.  Add rules of
the form 
\begin{verbatim}
    A_ij:- not not_A_ij
    not_A_ij:- not A_ij
\end{verbatim}
Clearly the 3-sat problem can be linearly translated to $P$.
