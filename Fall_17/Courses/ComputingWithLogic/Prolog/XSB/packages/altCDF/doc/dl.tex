\subsection{Type-0 CDF Instances and and Description Logics}
\label{sec:comp} 

Clearly Type-1 CDF instances are at least as expressive as ${\cal
ALCQ}$ description logics, since any ${\cal ALCQ}$ class expression
can be used directly in a {\tt necessCond/2} fact.  To formalize the
relation between Type-0 class expressions and description logics, we
use the definition of an {\em atomic relational CDF instance}.
\begin{definition}
Let $\cO$ be a CDF intance.  $\cO$ is {\em atomic} if it contains no
product identifiers.  $\cO$ is a {\em relational instance} if it
contains no object identifiers, and no {\tt classHasAttr/3}
predicates.
\end{definition}

We also need to formalize the language of class expressions that we
will use.

\begin{definition}
The syntax of an \omsdl{} class expression has the following form, in
which $A$ is an atomic class identifier, $R$ a relation identifer, $N$
a non-negative integer, and $C_1$ and $C_2$ \omsdl{} class expressions.
\[ C \leftarrow A | C_1 \sqcap C_2 | all(R,C_1) | exists(R,C_1) 
	| atLeast(N,R,C) | atMost(N,R,C) \]
\end{definition}

%--------------------------------------------------------------------------
\mycomment{
If $C$ is an \omsdl{} class expression, it can be translated according
to Definition~\ref{def:fot}, into a first order sentence (denoted as
$C^{\cT}$) over an ontology language.}

\mycomment{
TLS: dont think this is needed anymore, but check: 
These languages contain slightly different predicate
symbols from $\cL$, so we make use of a function $f$ from structures
over ontology languages to structures over first-order description
logic languages~\footnote{Given a structure $\cM$ over an ontology
language, $\cL_{CDF}$, we construct a language $f(\cL_{CDF})$ by
setting the set of atomic class names in $f(\cL_{CDF})$ equal to the
class identifiers in $\cL_{CDF}$, and the atomic relation names in
$\cL'_D$ equal to the relation identifiers in $\cL_{CDF}$.  A new
structure $f(\cM)$ is constructed by restricting $\cM$ to $elt/2$
and $rel/3$ predicates.}.
}
%--------------------------------------------------------------------------
We turn to a closer view on the relation between models of class
expressions and those of Type-0 instances.  A model $\cM$ is
$C$-reified for a class expression $C$ if for each sub-expression $C'$
of $C$,
\begin{itemize}
\item there is a constant $c'$ such that $isClass(c')$;  and 
\item $\cM \models (\exists X)[elt(X,c')]$;  and 
\item $C'^{\cI}[d]$ holds for all $d$ such that $\cM \models elt(d,c)$
\end{itemize}
$c'$ is called a witnessing constant for $C$.

In the following theorem, $TH(\cO)/4$ denotes $TH(\cO)$ minus Core
Axiom~\ref{ax:contained} (Domain Containment).  
\begin{theorem} \label{thm:type0dl}
Let $C$ be an \omsdl{} class expression.  Then 
\begin{enumerate}
\item There exists a $C$-reified model $f(\cM)$ for $C$.  
\item There is an atomic CDF instance $\cO$ such that 
\begin{enumerate}
\item For any $C$-reified model $f(\cM)$ of $C$, there is a CDF instance
$\cO$ such that $\cM \models TH(\cO)/4$
\item For any model $\cM$ of $TH(\cO)/4$, $f(\cM) \models (\exists X)
				[C^{\cI}[X]] $.
\end{enumerate}
\end{enumerate}
\end{theorem}
\begin{proof}
The proof is contained in the appendix.
\end{proof}

Atomic Type-0 CDF class instances thus have an expressive power that
is equivalent to a weak description logic, and be considered as
``special-cases'' of class descriptions that can be extended into
Type-1 instancs if needed, but otherwise are efficient for consistency
checking, subsumption checking and other operations.  

Due to Theorem~\ref{thm:type0dl}, we sometimes call an \omsdl{} class
expression a {\em Type-0 class expression}.

%-----------------------------------------------------------------------------------
\mycomment{
However, CDF can be compared to FLORA facts, for which it is somewhat
simpler.  CDF does not provide for a constraint that a given attribute
is functional as is allowed in FLORA, although this extension can be
provided if CDF instances are extended to have unqualified number
restrictions

Also, CDF does not provide for non-monotonic inheritance, but
substitutes the monotonic inheritances described in
Section~\ref{sec:inheritance}.  }
